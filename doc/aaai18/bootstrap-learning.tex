\def\year{2017}\relax
%File: formatting-instruction.tex
\documentclass[letterpaper]{article} %DO NOT CHANGE THIS
\usepackage{aaai17}  %Required
\usepackage{times}  %Required
\usepackage{helvet}  %Required
\usepackage{courier}  %Required
\usepackage{url}  %Required
\usepackage{graphicx}  %Required
\frenchspacing  %Required
\setlength{\pdfpagewidth}{8.5in}  %Required
\setlength{\pdfpageheight}{11in}  %Required
%PDF Info Is Required:
  \pdfinfo{}
\setcounter{secnumdepth}{0}

\usepackage{amsmath}
\usepackage{amssymb}
\usepackage{amsthm}
\usepackage{multirow}
\usepackage{tikz}
\usepackage{comment}

\usepackage{graphicx}
\usepackage{caption}
\usepackage{subcaption}
\usepackage{listings}
\usepackage{multicol}
\usepackage{arydshln}

\newcommand{\tup}[1]{{\langle #1 \rangle}}

\newcommand{\pre}{\mathsf{pre}}     % precondition
\newcommand{\del}{\mathsf{del}}     % effect
\newcommand{\add}{\mathsf{add}}     % effect
\newcommand{\eff}{\mathsf{eff}}     % effect
\newcommand{\cond}{\mathsf{cond}}   % conditional effect
\newcommand{\true}{\mathsf{true}}   % true
\newcommand{\false}{\mathsf{false}} % false
\newcommand{\PE}{\mathrm{PE}}     % precondition
\newcommand{\strips}{\textsc{Strips}}     % precondition

\newtheorem{theorem}{Theorem}
\newtheorem{lemma}[theorem]{Lemma}
\newtheorem{definition}[theorem]{Definition}


\begin{document}

\title{Bootstrapping classical planning action models}

\author{Guillem Franc\'es\\
{\small Information and Communication Technologies}\\
{\small Universitat Pompeu Fabra}\\
{\small Roc Boronat 138, 08018 Barcelona, Spain}\\
{\small @upf.edu}\\
\And Sergio Jim\'enez\\
{\small Computing and Information Systems}\\
{\small University of Melbourne}\\
{\small Parkville, Victoria 3010, Australia}\\
{\small sjimenez@unimelb.edu.au}\\
\And Nir Lipovetzky\\
{\small Computing and Information Systems}\\
{\small University of Melbourne}\\
{\small Parkville, Victoria 3010, Australia}\\
{\small @unimelb.edu.au}\\
\And Miguel Ram\'irez\\
{\small Computing and Information Systems}\\
{\small University of Melbourne}\\
{\small Parkville, Victoria 3010, Australia}\\
{\small @unimelb.edu.au}\\
}

\maketitle
\begin{abstract}
This paper presents a novel approach for learning classical planning action models from minimal input knowledge and using exclusively existing classical planners. First, the paper defines a classical planning compilation to learn action models from examples. Second, the paper explains how to collect informative examples using a classical planner based on pure exploratory search. 
\end{abstract}


\section{Introduction}
Off-the-shelf planners reason about action models that correctly and completely capture the possible world transitions~\cite{geffner:book:2013}. Unfortunately building such models is complex even for planning experts limiting the potential of automated planning~\cite{kambhampati:modellite:AAAI2007}. 

In Machine Learning (ML) models are not hand-coded but computed from examples~\cite{michalski2013machine}. Applying off-the-shelf ML to learning planning action models is not straightforward though. On the one hand the inputs of ML algorithms are usually objects features encoded as a finite set of values. Planning tasks include actions and are closer to the representation of procedures and behaviours than to object representation. On the other hand, the traditional output of off-the-shelf ML techniques is a scalar value (an integer, in the case of classification tasks, or a real value, in the case of regression tasks). For the task of learning classical planning actions the output is not a scalar but a declarative generative model of the state transitions. Finally the collection of {\em informative} examples for learning planning action models is complex. Planning actions include preconditions that are only satisfied by specific sequences of actions and often with a low probability of being chosen by chance~\cite{fern2004learning}.

The learning of action models for classical planners is a well-studied problem and there are sophisticated algorithms for this learning task, like ARMS~\cite{yang2007learning} or LOCM~\cite{cresswell2013acquiring} that do not require the full knowledge of the states traversed by example plans. This work aims going one step beyond these approaches and address learning classical planning action models when the only available knowledge is pairs of initial and final states.

Motivated by recent advances on learning generative models using classical planners~\cite{segovia2017generating} and on the effective exploration of planning state spaces~\cite{geffner:novelty:IJCAI17}, this paper introduces an innovative approach for learning classical planning action models. The contribution of the work is two-fold:
\begin{enumerate}
\item An inductive learning algorithm that minimizes the required input knowledge, that is flexible to different levels of available input knowledge, and that can be defined as a classical planning compilation. 
\item A method for autonomously collect {\em informative} examples for action model learning using an exploration-based classical planner.
\end{enumerate}


\section{Background}
This section defines the planning models used on this work.

\subsection{Classical Planning}
We use $F$ to denote the set of {\em fluents} (propositional variables) describing a state. A {\em literal} $l$ is a valuation of a fluent $f\in F$, i.e.~$l=f$ or $l=\neg f$. A set of literals $L$ represents a partial assignment of values to fluents (WLOG we assume that $L$ does not assign conflicting values to any fluent). We use $\mathcal{L}(F)$ to denote the set of all literal sets on $F$, i.e.~all partial assignments of values to fluents. A {\em state} $s$ is a total assignment of values to fluents, i.e. $|s|=|F|$, so the number of states is $2^{|F|}$. 

Under this formalism, a {\em classical planning frame} is a tuple $\Phi=\tup{F,A}$, where $F$ is a set of fluents and $A$ is a set of actions. Each action $a\in A$ has a set of literals $\pre(a)\in\mathcal{L}(F)$ called the {\em preconditions}, a set of positive effects $\add(a)\in\mathcal{L}(F)$ and a set of negative effects $\del(a)\in\mathcal{L}(F)$. An action $a\in A$ is applicable in state $s$ iff $\pre(a)\subseteq s$, and the result of applying $a$ in $s$ is a new state $\theta(s,a)=(s\setminus \neg\del(a))\cup\add(a)$.

Given a planning frame $\Phi=\tup{F,A}$, a {\em classical planning problem} is a tuple $P=\tup{F,A,I,G}$, where $I$ is an initial state and $G\in\mathcal{L}(F)$ is a goal condition. A {\em plan} for $P$ is an action sequence $\pi=\tup{a_1, \ldots, a_n}$ that induces a state sequence $\tup{s_0, s_1, \ldots, s_n}$ such that $s_0=I$ and, for each $i$ such that {\small $1\leq i\leq n$}, $a_i$ is applicable in $s_{i-1}$ and generates the successor state $s_i=\theta(s_{i-1},a_i)$. The plan $\pi$ {\em solves} $P$ if and only if $G\subseteq s_n$, i.e.~if the goal condition is satisfied following the application of $\pi$ in $I$.

We assume that fluents are instantiated from predicates, as in PDDL~\cite{fox2003pddl2}. Specifically, there exists a set of predicates $\Psi$, each $p\in\Psi$ with an argument list of arity $ar(p)$. Given a set of objects $\Omega$, the set of fluents $F$ is then induced by assigning objects in $\Omega$ to the arguments of predicates in $\Psi$, i.e.~$F=\{p(\omega):p\in\Psi,\omega\in\Omega^{ar(p)}\}$ where, given a set $X$, $X^n$ is the $n$-th Cartesian power of $X$.

Likewise we assume actions in $A$ are instantiated from operator schema as follows. Let us define $\Omega_v=\{var_1,\ldots,var_v\}$ as a new set of objects, $\Omega_v\cap\Omega=\emptyset$, representing variable names. The number of {\em variable} objects, $|\Omega_v|$, is given by the action with the maximum arity. For instance, in the blocksworld $\Omega_v=\{var_1,var_2\}$ since action $stack$, Figure~\ref{fig:stack}, (and $unstack$) has two parameters. Let us define a new set of fluents, $F_{v}\cap F=\emptyset$, that result instantiating $\Psi$ but using only the variable objects in $\Omega_v$.

\begin{figure}[hbt]
\begin{footnotesize}
\begin{verbatim}
  (:action stack
   :parameters (?x1 ?x2)
   :precondition (and (holding ?x1) 
                      (clear ?x2))
   :effect (and (not (holding ?x1))
                (not (clear ?x2))
                (clear ?x1)
                (handempty)
                (on ?x1 ?x2)))
\end{verbatim}
\end{footnotesize}
 \caption{\small Example of the {\em stack} planning action schema from the blocksworld as represented in PDDL.}
\label{fig:stack}
\end{figure}

Now we are ready to define an operator schema $\xi\in \Xi$ as:
\begin{itemize}
\item The operator {\em header}: a fluent $header(name(\xi),pars(\xi))$ built using the action name, e.g. $name(\xi)\in\{stack,unstack,pickup,putdown\}$ for the blocksworld, and a list of variables, $pars(\xi)\in\Omega_v^{ar(\xi)}$. 
\item The operator {\em body} comprises: the {\em preconditions}, $pre(\xi)\subseteq F_v$, the {\em positive effects}, $add(\xi)\subseteq F_v$, and the {\em negative effects}, $del(\xi)\subseteq F_v$ such that, $del(\xi)\subseteq pre(\xi)$, $del(\xi)\cap add(\xi)=\emptyset$ and $pre(\xi)\cap add(\xi)=\emptyset$.
\end{itemize}


\subsection{Classical Planning with Conditional Effects}
Conditional effects make it possible to repeatedly refer to the same action even though their precise effects depend on the current state. In this case each action $a\in A$ has a set of literals $\pre(a)\in\mathcal{L}(F)$ called the {\em precondition} and a set of conditional effects $\cond(a)$. Each conditional effect $C\rhd E\in\cond(a)$ is composed of two sets of literals $C\in\mathcal{L}(F)$ (the condition) and $E\in\mathcal{L}(F)$ (the effect). An action $a\in A$ is applicable in state $s$ if and only if $\pre(a)\subseteq s$, and the resulting set of {\em triggered effects} is
\[
\eff(s,a)=\bigcup_{C\rhd E\in\cond(a),C\subseteq s} E,
\]
i.e.~effects whose conditions hold in $s$. The result of applying $a$ in $s$ is a new state $\theta(s,a)=(s\setminus \neg\eff(s,a))\cup\eff(s,a)$.


\section{Learning classical planning action models}
This section formalizes the different learning tasks addressed in the paper according to the available amount of input knowledge.

The largest possible amount of available input knowledge is learning from the {\em pre-} and {\em post-states} of every action execution. This learning task is straightforward since the dynamics of a given operator are derived lifting the literals that change between the pre and post-state of the corresponding action execution. The preconditions are the minimal set of literals appearing in all the pre-states corresponding to the same operator.

\subsubsection{Learning classical planning action models from plans}
This task is formalized as $\Lambda=\tup{\Psi,\Pi,\Sigma}$: 
\begin{itemize}
\item $\Psi$ the set of predicates that define the abstract state space of a given planning domain. This set includes the predicates describing the actions header.
\item $\Pi=\{\pi_1,\ldots,\pi_t\}$ is the given set of example plans,
\item $\Sigma=\{\sigma_1,,\ldots,\sigma_t\}$ is a set of labels s.t. each plan $\pi_i$, {\small $1\leq i\leq t$}, has an associated label $\sigma_i=(s_i,s_i')$ where $s_i'$ is the state resulting from executing $\pi_i$ starting from the state $s_i$. 
\end{itemize}

A solution to the learning task $\Lambda$ is a set of operator schema $\Xi$ that is compliant with the predicates in $\Psi$, the example plans $\Pi$, and their labels $\Sigma$.

%\subsubsection{Learning classical planning action models from partially specified plans}

\subsubsection{Learning classical planning action models from states}

As explained in this paper we aim to reduce the amount of input knowledge provided to the learning task so we redefine it as $\Lambda'=\tup{\Psi,\Sigma}$.

A solution to the $\Lambda'$ learning task is a set of operator schema $\Xi$ that is compliant with the predicates in $\Psi$, and the given set of initial and final states.


\section{Learning action models from states using a classical planner}
Our approach for addressing $\Lambda'=\tup{\Psi,\Sigma}$ is compiling this learning task into a classical planning task $P_{\Lambda}$. The intuition behind the compilation is that a solution to $P_{\Lambda}$ is a sequence of actions that first, programs the action action model (i.e. the $pre(\xi)$, $del(\xi)$ and $add(\xi)$ for each $\xi\in\Xi$) and then, sequentially, validates the programmed action model in the given set of labels $\Sigma$, one after the other.

To formalize the compilation we first define $t$ classical planning instances $P_1=\tup{F,A,I_1,G_1},\ldots,P_t=\tup{F,A,I_t,G_t}$, that belong to the same planning frame $\Phi=\tup{F,A}$ (i.e. share the same fluents and actions and differ only in the initial state and goals). The sets of actions is empty $A=\emptyset$ while the set of fluents $F$ is built instantiating the predicates in $\Psi$ with the objects in $\Omega$, where $\Omega$, is the set of objects that appear in the states in $\Sigma$, i.e., $\Omega=\{o|o\in s_i\cup s_i', {\small 1\leq i\leq t}\}$. Finally the initial state $I_i$, {\small $1\leq i\leq t$}, is given by the state $s_i\in \sigma_i$ and the goals are defined by the state $s_i'\in \sigma_i$. 

Now we are ready to define the compilation for learning action models using an off-the-shelf classical planner. Given a learning task $\Lambda'=\tup{\Psi,\Sigma}$ the compilation outputs a classical planning task $P_{\Lambda}=\tup{F_{\Lambda},A_{\Lambda},I_{\Lambda},G_{\Lambda}}$:
\begin{itemize}
\item $F_{\Lambda}$ extends $F$ with:
\begin{itemize}
\item Fluents representing the programmed action model: $header(\xi)$, $pre_f(\xi)$, $del_f(\xi)$ and $add_f(\xi)$ where $f\in F_v$ and $\xi \in \Xi$.
\item Fluents $\{test_i\}_{1\leq i\leq t}$, indicating the example where the programmed model is currently being validated.
\end{itemize}
\item $I_{\Lambda}$, contains the fluents from $F$ that encode the initial state $s_1\in P_1$ and the header of the operators we want to learn.
\item $G_{\Lambda}=\{test_i\}$,{\small $1\leq i\leq t$}, indicates that the programmed action model is validated in all the examples.
\item $A_{\Lambda}$ replaces the actions in $A$ with actions of three types:
\begin{enumerate}
\item Actions for programming:
\begin{itemize}
\item A {\em precondition} $f\in F_v$ in the action schema $\xi\in\Xi$:
\begin{small}
\begin{align*}
\hspace*{7pt}\pre(\mathsf{programPre_{\xi,f}})=&\{\neg pre_{f}(\xi),\neg del_{f}(\xi),\neg add_{f}(\xi)\},\\    
\cond(\mathsf{programPre_{\xi,f}})=&\{\emptyset\}\rhd\{pre_{f}(\xi)\}.
\end{align*}
\end{small}
\item A {\em negative effect} $f\in F_v$ in the action schema $\xi\in\Xi$:
\begin{small}
\begin{align*}
\hspace*{7pt}\pre(\mathsf{programDel_{\xi,f}})=&\{pre_{f}(\xi),\neg del_{f}(\xi),\neg add_{f}(\xi)\},\\                                                   
\cond(\mathsf{programDel_{\xi,f}})=&\{\emptyset\}\rhd\{del_{f}(\xi)\}.
\end{align*}
\end{small}

\item A {\em positive effect} $f\in F_v$ in the action schema $\xi\in\Xi$:
\begin{small}
\begin{align*}
\hspace*{7pt}\pre(\mathsf{programAdd_{\xi,f}})=&\{\neg pre_{f}(\xi)),\neg del_{f}(\xi)),\neg add_{f}(\xi)\},\\                                                   
\cond(\mathsf{programAdd_{\xi,f}})=&\{\emptyset\}\rhd\{add_{f}(\xi)\}.
\end{align*}
\end{small}
\end{itemize}

\item Actions for applying an already programmed operator schema $\xi\in\Xi$ and that is bound with objects $\upsilon'\in\Omega^{ar(\xi)}$)
\begin{small}
\begin{align*}
\hspace*{7pt}\pre(\mathsf{apply_{\xi,\upsilon,\upsilon'}})=&\{header(\xi)\}\cup\\     
                                           &\{pre_{f}(\xi)\implies p(\upsilon')\}_{\forall p\in\Psi},\\
\cond(\mathsf{apply_{\xi,\upsilon,\upsilon'}})=&\{del_{f}(\xi)\}\rhd\{\neg p(\upsilon')\}_{\forall p\in\Psi},\\
&\{add_{f}(\xi)\}\rhd\{p(\upsilon')\}_{\forall p\in\Psi}.
\end{align*}
\end{small}

\item Actions for changing the active example where the action model is currently being validated.
\begin{small}
\begin{align*}
\hspace*{7pt}\pre(\mathsf{validate_{i}})=&G_i\cup\{test_j\}_{j\in 1\leq j<i},\\
\cond(\mathsf{validate_{i}})=&\{\emptyset\}\rhd\{test_i\}.
\end{align*}
\end{small}
\end{enumerate}
\end{itemize}


\begin{lemma}
Any classical plan $\pi$ that solves $P_{\Lambda}$ induces a valid action model that solves the learning task $\Lambda$.
\end{lemma}

\begin{proof}[Proof sketch]
Once an operator schema is programmed it cannot be modified and can only be executed. The only way of achieving a test fluent is by applying a sequence of programmed operator schema until achieving the goal state in its associated label. If this is done for all the input examples it means that the programmed action model is compliant with the learning input knowledge and hence, that solves the action model learning task.
\end{proof}

Interestingly this compilation is valid for partially specified action models since the known preconditions and effects (fluents $pre_f(\xi)$, $del_f(\xi)$ and $add_f(\xi)$) can be part of the initial state $I_{\Lambda}$. With this regard the approach allows also transfer learning in which we generate the action model for a given sub-task and then encode this model as already programmed actions for learning new or more challenging action models.

\section{Learning action models from plans using a classical planner}
The compilation can be extended to the learning scenario in which a set of action plans $\Pi$ is available s.t., each plan $\pi_i\in \Pi$, {\small $1\leq i\leq t$}, is a solution to the corresponding classical planning instance $P_i=\tup{F,A,I_i,G_i}$:
\begin{itemize}
\item $F_{\Lambda}$ includes the set of fluents $F_{plan}=\{plan(name(\xi),i,\Omega^{ar(a)})\}$ for encoding the plans in $\Pi$. Fluents $at_i$ and $next_{i,i_2}$, {\small $1\leq i<i2\leq n$}, indicate the plan step where the programmed action model is being validated and $n$ is the max length of a plan in $\Pi$.
\item $I_{\Lambda}$ is extended with the fluents from $F_{plan}$ that encode the plan $\pi_1\in \Pi$ for solving $P_1$, and the fluents $at_1$ and $\{next_{i,i_2}\}$, {\small $1\leq i<i2\leq n$}, for indicating the plan step where to start validating the programmed model. Goals are like in the original compilation.
\item With respect to the actions,
\begin{enumerate}
\item The actions for programming the preconditions and effects of a given operator schema are the same.
\item The actions for applying an operator schema have an extra precondition $f\in F_{plan}$ s.t. $f$ encodes the current plan step and extra conditional effect $\{at_{i}\}\rhd\{\neg at_{i},at_{i+1}\}_{\forall i\in 1\leq i< n}$ for advancing the plan step.
\item The actions for changing the active test have an extra precondition, $at_{|\Pi_i|}$, to indicate that we simulated the full current plan and extra conditional effects to load the next plan where to validate the programmed action model:
\begin{small}
\begin{align*}
&\{\emptyset\}\rhd\{\neg plan(\xi,k,\upsilon\}_{1\leq k<|\Pi_i|},\\
&\{\emptyset\}\rhd\{plan(\xi,k,\upsilon\}_{1\leq k<|\Pi_{i+1}|},\\
&\{\emptyset\}\rhd\{\neg at_{|\pi_i|},at_1\}.
\end{align*}
\end{small}
\end{enumerate}
\end{itemize}



\section{Generating informative learning examples}
Observation: If a predicate is not appearing either in the initial not in the final state it will not appear in the learned action model. 


\section{Evaluation}
We evaluate the performance of our learning approach using the cardinality of the {\em symmetric difference} sets that are computed between the set of preconditions, del and add effects, in the learned model and in the actual models. This performance metric is weighted by the size of the sets so a $100\%$ value means that this set was learned perfectly and $0\%$ means that no correct learning was accomplished.

The performance of our learning approach is evaluated for (1) different amounts of initial knowledge available and (2) using different sources for collecting this initial knowledge.

\begin{tabular}{l|cc|cc|cc}
 & \multicolumn{2}{c|}{Prec} & \multicolumn{2}{c|}{Del} & \multicolumn{2}{c}{Add} \\\hline
& Avg & Var & Avg & Var & Avg & Var\\\hline
Blocks & 5.0 & 75.0 & 0.0 & 0.0 & 0.0 & 0.0\\
Grippper& 65.5 & 150.6 & 0.0 & 0.0 & 5.1 & 14.4 \\
Miconic& 41.3 & 482.8 & 0.0 & 0.0 & 6.3 & 41.5\\
Visitall& 50.0 & 0.0 & 0.0 & 0.0 & 0.0 & 0.0
\end{tabular}


\section{Related work}

%The  LIVE  system  (Shen  and  Simon,  1989)  was  an  extension  of  the  General  Problem  Solver  (GPS) framework (Ernst and Newell, 1969) with a learning component. LIVE alternated problem solving with model  learning  to  automatically  define  operators.  The  decision  about  when  to  alternate  depended  onsurprises,  that  is  situations  where  an  action  effects  violated  its  predicted  model.  EXPO  (Gil,  1992) generated plans with the  PRODIGY  system (Minton, 1988), monitored the plans execution, detected differences in the predicted and the observed states and constructed a set of specific hypotheses to fix those differences. Then the  EXPO  filtered the hypotheses heuristically.  OBSERVER  (Wang, 1994) learned operators by monitoring expert agents and applying the version spaces algorithm (Mitchell, 1997) to the observations. When the system already had an operator representation, the preconditions were updated by removing facts that were not present in the new observation’s pre-state; the effects were augmented by adding facts that were in the observation’s delta-state. All of these early works were based on direct liftings of the observed states. They also benefit from experience beyond simple interaction with the environment such as exploratory plans or external teachers, but  none  provided  a  theoretical  justification  for  this  second  source  of  knowledge.  The  work  recently reported in (Walsh and Littman, 2008) succeeds in bounding the number of interactions the learner must complete to learn the preconditions and effects of a STRIPS action model. This work shows that learning STRIPS  operators  from  pure  interaction  with  the  environment,  can  require  an  exponential  number  of samples, but that limiting the size of the precondition lists enable sample-efficient learning (polynomial in the number of actions and predicates of the domain). The work also proves that efficient learning is also possible without this limit if an agent has access to an external teacher that can provide solution traces on demand.

%Others systems have tried to learn more expressive action models for deterministic planning in fully observable environments. Examples would include the learning of conditional costs for AP actions (Jess Lanchas and Borrajo, 2007) or the learning of conditional effects with quantifiers (Zhuo et al., 2008).

%In addition action model learning has been studied in domains where there is partial state observability. ARMS uses the same kind od learning examples but assumes the exampls are given and proceeds in two phases. In the first phase, ARMS extracts frequent action sets from plans that share a common set of parameters. ARMS also finds some frequent literal-action pairs with the help of the initial state and the goal state that provide an initial guess on the actions preconditions, and effects. In the second phase, ARMS uses the frequent action sets and literal-action pairs to define a set of weighted constraints that must hold in order to make the plans correct. Then, ARMS solves the resulting weighted MAX-SAT problem and produces action models from the solution of the SAT problem. This process iterates until all actions are modelled. For a complex planning domain that involves hundreds of literals and actions, the corresponding weighted MAX-SAT representation is likely to be too large to be solved efficiently as the number of clauses can reach up to tens of thousands. For that reason ARMS implements a hill-climbing method that models the actions approximately. Consequently, the ARMS output is a model which may be inconsistent with the examples.

%(Amir and Chang, 2008) introduced an algorithm that tractably generates all the STRIPS-like models that  could  have  lead  to  a  set  of  observations.  Given  a  formula  representing  the  initial  belief  state,  a sequence of executed actions and the corresponding observed states(where partial observations of states are given), it builds a complete explanation of observations by models of actions through a Conjunctive Normal Form (CNF) formula. By linking the possible states of fluents to the effect propositions in the action models, the complexity of the CNF encoding can be controlled to find  exact  solutions  efficiently  in  some  circumstances.  The  learning  algorithm  updates  the  formula  of the belief state with every action and observation in the sequence. This update makes sure that the new formula represents all the transition relations consistent with the actions and observations. The formula returned at the end includes all consistent models, which can then be retrieved with additional processing. Unlike the previous approaches, the one described in (Mour ao et al., 2008) deals with both missing and noisy predicates in the observations. For each action in a given domain, they use kernel perceptrons to learn predictions of the domain properties that change because of the action execution. LOCM (Cresswellet al., 2009) induces action schemas without being provided with any information about initial, goal or intermediate state descriptions for the example action sequences. LOCM receives descriptions of plans or plan fragments, uses them to create states machines for the different domain objects and extracts the action schemas from these state machines.

\section{Conclusions}

Learning action models from examples allows the reformulation of a domain theory. Using classical planning focus the learning task in the more succinct action models given that classical planners have a preference for short plans.

\bibliographystyle{aaai}
\bibliography{bootstrap-learning}
\end{document}
